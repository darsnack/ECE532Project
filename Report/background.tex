The Singular Value Decomposition (SVD) is a popular mathematical tool, with applications in linear algebra, signal processing, image processing, statistics, and control theory. The SVD factorizes an $m\times n$ complex matrix of rank-$r$, $A$, into the form:

\[ A = U \Sigma V^\intercal \]

Where $U$ is an $m\times m$ unitary matrix and $V$ is an $n \times n$ unitary matrix. $\Sigma$ is an $m \times n$ rectangular matrix, which has only $r$ nonzero elements (called ``singular values'') on its diagonal, which are traditionally arranged in descending order. All other elements of $\Sigma$ are zero. So we assume $\sigma_1 \geq \sigma_2 \geq \ldots \geq \sigma_r > 0$. As expected, $r = \mathrm{rank}(A) \leq \min(m,n)$.

We call the columns of $U$, $[u_1, u_2, \ldots, u_m]$ the left singular vectors of $A$. Similarly, we refer to the columns of $V$, $[v_1, v_2, \ldots, v_n]$ as the right singular vectors of $A$.

In this lab, we will also refer to the ``economy'' SVD. This version of the SVD factorizes $A$ as:

\[ A = U_1 \Sigma_1 V_1^\intercal \]

Where $\Sigma_1$ is an $r \times r$ diagonal matrix containing only positive singular value elements. $U_1$ is an $m \times r$ unitary matrix containing only the first $r$ left singular vectors of $A$, and $V_1$ is an $n \times r$ unitary matrix containing only the first $r$ right singular vectors of $A$. This decomposition is smaller than the full SVD, and so can be useful when dealing with extremely large matrices.

Now we can represent $A$ as a linear combination of rank-1 matrices:

\[ A = \sigma_1 u_1 v_1^\intercal + \sigma_2 u_2 v_2^\intercal + \ldots + \sigma_r u_r v_r^\intercal \]

From class, you will remember the Eckart-Young Theorem\cite{Eckart1936}:

\begin{theorem}
	Given a matrix, $A \in \set{R}^{m \times n}$ that has rank $r$, we can find a rank $k \leq r$ approximation to $A$, $X$ as
	$$\sum_{i = 1}^k \sigma_i u_i v_i^\top$$
	where $\sigma_i$ are the singular values of $A$, and $u_i$ and $v_i$ are the singular vectors of $A$. Moreover, $X$ is  the solution to the problem
	$$\arg \min_{\rank{B} \leq k} \|A - B\|$$
	for any unitarily-invariant norm.
	\label{thm:eckart-young}
\end{theorem}

This tells us that the SVD can be used to find the best low-rank approximation possible. In particular, these low-rank methods can be used for image processing. 

There are many interesting applications of the SVD in image processing, including several related to Principal Component Analysis (PCA) and advanced filtering techniques \cite{1162766}\cite{1326538}\cite{382496}. The SVD can also be used on video and image data to reconstruct fragmented or damaged photos and videos \cite{VideoRest}.

In this lab, we will first explore some example image processing techniques, particularly low-rank approximations. Afterwards, we will investigate how to structure video data to be analyzable using our methods, and explore the significance of a low-rank video approximation. Finally, we will conclude with a video editing scheme based entirely around the SVD. Let's start with the warm-up!